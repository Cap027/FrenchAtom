\include{preamble_base.tex}

\begin{document}
	
\section{GCR}

Первые 8 (9) реакторов Франции были газоохлаждаемого типа проекта \textbf{UNGG} "--- Uranium Naturel Graphite Gaz. Несмотря на то, что все реакторы формально принадлежат одному проекту, каждый из них имеет множество конструктивных особенностей.\\
\textbf{Замедлитель} "--- графит,\\
\textbf{Теплоноситель} "--- углекислый газ,\\
\textbf{Топливо} "--- природный металлический уран.

Вопреки расхожему мнению, проект \textbf{UNGG} развивался \textbf{независимо} от британского проекта \textbf{Магнокс}, несмотря на аналогичные требования.

Первый реактор \textbf{Marcoule G-1} обычно не включается в список энергоблоков, несмотря на то, что подключался в сеть и имел мощность в 2~МВт. Прямое назначение данного реактора "--- \textbf{наработка плутония}, а будучи включенным в сеть, реактор потреблял больше электроэнергии, чем был способен выдать.
	
\section{HWGCR}

\textbf{АЭС Бреннилис} имеет один энергоблок с тяжеловодным газоохлаждаемым реактором \textbf{(HWGCR)} мощностью 75~МВт.

\textbf{Введена} в эксплуатацию в \textbf{1967 году}, \textbf{выведена} из эксплуатации в \textbf{1985}. Причиной закрытия стала смена направления французской атомной энергетики \textbf{в пользу реакторов PWR}.

15 августа 1975 года на станцию была произведена \textbf{террористическая атака}. Два взрыва слегка повредили турбину и уничтожили линию телеграфа. Ответственность взяла на себя радикальная группировка \textbf{<<Фронт освобождения Бретани>>}. В \textbf{1979} году данная группировка \textbf{уничтожила линию электропередачи}, соединяющую АЭС с сетью, в связи с чем \textbf{реактор был остановлен}. По сей день это единственный в истории случай, когда террористической группировке удалось остановить работу АЭС.

Стоимость \textbf{вывода} АЭС Бреннилис из эксплуатации обошлась в \textbf{482 млн. евро}, что на много превысило ожидания.

\section{PWR}

\textbf{Нефтяной кризис 1974} г. привел французские власти к пониманию
того факта, что \textbf{только атомная} энергетика способна обезопасить
страну от подобных кризисов. Для реализации больших объемов выработки электроэнергии требовался \textbf{надежный и простой в эксплуатации} реактор. На этот момент у страны уже был опыт эксплуатации первого во франции PWR на АЭС ШО, что и определило выбор технологии \textbf{PWR} как основной для дальнейшего развития.

По всей стране строятся реакторы типов \textbf{CP0, CP1 и CP2} мощностью в \textbf{900 МВт} (исключением является лишь п\textbf{ервый PWR на АЭС ШО мощностью 320 МВт}).

Реакторы типа \textbf{PWR} "--- \textbf{единственные} реакторы, которые продолжают работать на \textbf{данный момент}. \textbf{Всего} в эксплуатации находятся \textbf{56} реакторов и \textbf{1} "--- в процессе \textbf{строительства}, \textbf{3} реактора типа PWR (серии CP0) были \textbf{выведены} из эксплуатации.

\section{FBR}

На первых этапах развития реакторов на быстрых нейтронах перед проектировщиками стояло много вопросов, которые прежде никто не рассматривал. 

Для этих целей был спроектирован \textbf{экспериментальный реактор <<Рапсодия>>}. Была выбрана петлевая схема отвода тепла "--- это был оптимальный выбор для экспериментальной установки.

Строительство началось в 1962 году, и в этом же году реактор достиг критичности при номинальной мощности в 20 МВт-тепловых (далее "--- МВт(т)). В конце 1967 года его мощность была увеличена до 24 МВт(т), и в 1970 году, после доработки активной зоны, до 40 МВт(т).

Уже тогда стала очевидной главная проблема эксплуатации подобных ЯЭУ "--- течи высоко химически-активного натрия.

Опыт эксплуатации реактора <<Рапсодия>> окончательно установил, что компоновка энергетической установки \textbf{обязана быть интегральной}.

Мощность первого энергетического реактора \textbf{<<Феникс>>} выбиралась из двух вариантов: 600 и 1200 МВт (тепловая). Для начала решили ограничиться малым. Электрическая же мощность Феникса по проекту и вовсе \textbf{не превышала 250 МВт}, \textbf{фактически} за все время эксплуатации "--- \textbf{не поднималась выше 230}. Более того, с 1997 года после ряда происшествий было принято решение \textbf{ограничить} мощность до \textbf{130 МВт}. 

Частые неполадки и решение об ограничении мощности снизило \textbf{КИУМ} до рекордно низких \textbf{41 \%}.

Проблеме скачков реактивности стоит уделить отдельное внимание. Данная проблема получила название \textbf{A.U.R.N.} (фр. Arrêt d’urgence par réactivité négative) "--- автоматический аварийный останов по отрицательной реактивности. Всего было зафиксировано \textbf{4 события}: 6 августа, 24 августа и 14 сентября 1989 года и 9 сентября 1990 года, произошедших по одному и тому же сценарию:
\begin{enumerate}
	\item Почти линейное резкое увеличение отрицательной реактивности и, соответственно, уменьшение мощности. Всего за 50 мс мощность падала до 28--45 \% от начальной (в этот момент срабатывала аварийная защита).
	
	\item Симметричный резкий подъем мощности почти до начального значения.
	
	\item Снова падение, хотя и менее резкое и глубокое, через 150 мс после начала события.
	
	\item Вторичный пик, который слегка превосходит изначальную мощность реактора.
	
	\item Падение мощности в результате введения автоматикой в активную зону поглощающих стержней.
\end{enumerate}

Проблема так и не получила окончательного объяснения, наиболее правдоподобным считается объяснение с помощью явления, получившего название <<Core-flowering>> (англ. <<расцеветание>> (подобно бутону) активной зоны).

В связи с тем, что проект \textbf{<<Суперфеникс>>}, представлявший из себя масштабирование Феникса, развивался на фоне проблем не столь удачного предка, еще на этапе строительства новый проект стал объектом \textbf{политических и идеологических скандалов}.

В середине апреля 1976 года президент Валери Жискар д’Эстен и его энергетические советники приняли политическое решение о строительстве Superphenix.

Летом 1976 года на площадку проникло почти 20 тысяч демонстрантов, протестовавших против планов по строительству быстрого реактора. В период 1974—1976 годов в оппозиции проекту находилось около 50 местных муниципалитетов. Противостояние накалялось и нашло свой трагический выход в событиях 31 июля 1977 года. В этот день около 50 тысяч демонстрантов вступили в схватку с полицией. Один из протестующих был убит, двое потеряли руки и ноги.

18 января 1982 года стройплощадка была обстреляна с другого берега Роны из противотанкового РПГ-7. Пять ракет не нанесли существенного урона станции, но информационного масла в огонь подлили. Позже выяснилась причастность к атаке бельгийской леворадикальной организации \textbf{<<Cellules Communistes Combattantes>>}.

После включение станции в сеть первый серьёзный инцидент на Superphenix случился в мае 1987 года. Персонал обнаружил большую утечку натрия из бака системы обращения с топливом. Отремонтировать бак не удалось. Понадобилось 10 месяцев для разработки новой процедуры загрузки и выгрузки топливных кассет из активной зоны. После того, как была предложена новая процедура обращения с топливом, ушло ещё 13 месяцев на её квалификацию и получение разрешительных бумаг. Таким образом, Superphenix вернулся в строй только в апреле 1989 года.

После повторного пуска реактор работал на малых уровнях мощности. В июле 1990 года произошла новая беда "--- отказ компрессора привел к впрыску значительного объема воздуха в контур и окислению натрия. На очистку натрия ушло восемь месяцев.

А в декабре 1990 года после сильного снегопада обрушилась крыша машинного зала. Вопрос о запуске Superphenix стал предметом длительных парламентских слушаний и дебатов на национальном и региональном уровнях. В июне 1992 года правительство назначило новые общественные слушания на период 30 марта — 14 июня 1993 года. В январе 1994 года правительство получило отзыв от органов атомнадзора. В июле 1994 года, наконец, новая лицензия на эксплуатацию была выдана. Блок вернулся в строй — и проработал всего семь месяцев. Причиной очередного останова стала утечка аргона в теплообменник.

28 февраля 1997 года госсовет аннулировал эксплуатационную лицензию. 19 июня 1997 года премьер-министр Лионель Жоспен заявил: «Superphenix будет закрыт». 30 декабря 1998 года был подписан указ, оформивший решение о закрытии Superphenix.

В итоге \textbf{за 11 лет} после подключения к электросетям станция была \textbf{в работе 63 месяца}, в основном на малой мощности; 25 месяцев она была отключена по техническим причинам, а 66 месяцев — по политическим и административным.

\section{Будущее ядерной энергетики во Франции}

По стратегии президента \textbf{Эммануэль Макрона} в 2020 г. была закрыла \textbf{АЭС Фессенхайм} в Эльзасе, что сократило число АЭС во Франции до 18.

\textbf{Министр комплексных экологических преобразований Николя Юло} 10  июля \textbf{2017 г}. объявил о  планах закрытия
до \textbf{17} ядерных \textbf{реакторов} во исполнение закона о реформировании
энергетики, предусматривающего \textbf{снижение до 50 \% }доли атомной
энергии в общем объеме производства электроэнергии во Франции.

Относительно недавно Франция сделала ставку на \textbf{возрождение ядерной энергетики} после десяти лет раздумий, за которые многие страны Европы вслед за Германией закрыли почти все АЭС. Решиться на многомиллиардные траты президенту \textbf{Эммануэлю Макрону} помог энергетический кризис, вызванный \textbf{дефицитом российского газа}.

\section{Альтернативные источники энергии}

Во Франции планировалось к 2025 году сократить \textbf{долю АЭС} до \textbf{50 \%} за счёт прироста доли \textbf{ВИЭ} до\textbf{ 40 \%}, тогда как на долю \textbf{теплоэлектростанций} останутся те же \textbf{10 \%}.

Динамика ввода новых объектов по выработке электроэнергии из возобновляемых источников вселяет оптимизм в инвесторов. Планируется, что \textbf{солнечные станции} будут покрывать 5,4 тыс. МВт пиковой мощности. При этом уже в 2013 году установленная мощность солнечных панелей превышала 4 ГВт. В том же году министерство энергетики Франции анонсировало увеличение ежегодного прироста солнечных мощностей с 500 до 1 тыс. МВт.

В любом случае на ближайшую перспективу страна \textbf{не собирается отказываться} от использования ни от одного существующего ныне источника энергии. Речь, скорее всего, идёт об их рациональном \textbf{перераспределении} в общем энергобалансе страны.

\end{document}